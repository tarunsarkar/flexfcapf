% chap1.tex {Introduction}

\chapter{Introduction}\label{INTR-CHAP}
Earlier, people were using mobile devices mainly for making calls and sending text messages. Today, wireless devices have become a main tool for surfing the internet, staying connected with friends and family, playing games, watching live news and streaming videos \cite{6146489}. To accommodate this development, cloud computing and storage services have recently grown and in effect caused the increase in the use of internet multiple times. Due to the diversity of internet uses and mobile devices, as well as the surge in the number of mobile users, traffic demand on mobile networks is growing exponentially, both in terms of total volume and the rate of data each user requires \cite{icsssti}. 

To cope with this consistently increasing traffic demand in mobile access networks, modern cellular network deployments become more and more complex and heterogeneous, or also called DenseNets \cite{6146489}. Extending the underlying backhaul network may be very costly and the network complexity at different levels present new management challenges and flexibility requirements. Thus, an efficient management of such network is necessary, which is why efficiently controlling DenseNets is a hot topic in research today. The Software-Defined Networking (SDN) has been identified as a promising solution to tackle these challenges. SDN separates the control plane from the data plane, which are usually tied together in conventional network devices. This solution provides the operator the ability to manage, control and monitor network devices using custom applications and enables an increased network flexibility independent of proprietary solutions.

Moreover, future wireless access networks support a broad variety of network functions with different purposes. For example, techniques like Coordinated Multi-Point (CoMP) \cite{comp} transmission, over coordinating network mechanisms for Inter-Cell Interference Coordination (ICIC) and Virtual Network Functions (VNFs) have virtualized applications to gain flexibility and save hardware costs. Along with that, trends such as network function virtualization (NFV) \cite{DBLP:journals/corr/MijumbiSGBTB15} lead to a growing relevance of data flow processing in the network. The data flow processing in the network imposes many other requirements on network resources. Choosing the right location for data processing is a challenging task and important for network performance.

However, while SDN controllers support the routing of information flow, sophisticated processing mechanisms like NFV or CoMP are not easy to include in the controller. Recent work describes Flow processing-aware Control Application Placement Problem (FCAPP) \cite {7343600} \cite{aurouxew2017} for an efficient management of future wireless networks, which considers both the aforementioned situation and provides a solution to perform network control and data flow processing in the network.

\section{Motivation}
New networking technologies such as cloud services, mobile computing, and server virtualization are difficult to implement using the conventional network architecture. The main issue with conventional networking devices is they are operated by specialized control planes with specialized features on top of proprietary hardware. Only networking hardware vendors know how to make things work on them and the user has very limited control. Recently, many network equipment vendors have started supporting open source networking architecture, which is why SDN is currently of a high-interest area in the industry. Most of the organizations, including Cisco, Juniper Networks, Broadcom, Brocade, Dell, HP, IBM, Intel, Microsoft, NEC, and Google, are developing components and standards based on SDN. For example, Google has developed its own OpenFlow controller and deployed an OpenFlow-enabled SDN solution using commodity hardware in its data center network \cite{googlesdn}.

Performance evaluation is a critical part of the software development life cycle. The faster defects can be discovered and corrected, the more quickly and cost-effectively a product can be deployed. New SDN solutions need rigorous testing and evaluation before deploying in real production-ready networks. SDN is still an open research field and it is evolving continuously. New design decisions and protocol changes are very common at this stage. Therefore, a high-fidelity testbed for evaluating large-scale SDN application is necessary to facilitate the transformation of research efforts on SDN into real productions. The success or failure of any given testing environment is determined by how closely, efficiently and cost-effectively the target environment can be reproduced.

Researchers have created real SDN testbed such as OCEAN \cite{ocean} and 100G SDN \cite{100gsdn} to provide a realistic environment where users can conduct real networking experiments. However, it has its limitations. First, the user has very limited control and undergoes a long process of accessing the test environment. Second, the user has limited flexibility on scenarios with which he can experiment. Thus, researchers developed many simulations and emulation testbeds for better control, quick accessibility, and where test scenarios could be reproduced easily. Emulation testbeds such as Mininet \cite{min-intro} and MaxiNet \cite{max-over} utilize Linux virtualization technologies and offer better scalability and flexibility than the real testbed. SDN simulation testbed like NS-3 \cite{ns-3} offers more cost-effective, scalable option but the accuracy degrades because of the simplification and abstraction of the simulation model.

In this thesis, it is desired to develop a testbed that combines most of the advantages of simulation, emulation, and real testbed (see Section \ref{sec:nta}) to test and evaluate FCAPP solution. FCAPP addresses two main problems of today's network - combining data flow processing and assign or reassign controllers in the network. FCAPP follows an evolving process of solving this critical problem. In \cite{7136368}, the authors proposed an optimization model to solve the problem and prove it is an NP-hard problem while in \cite{7127739}, the authors proposed an efficient heuristic algorithm to solve the problem. Though the solution provided in \cite{7136368} and \cite{7127739} gave good results, the dynamic nature of traffic in real world network was not considered in those studies. It considered only the static snapshot of the network. In \cite{7343600}, the authors rectified this shortcoming (i.e., dynamic nature was considered) and proposed an improved solution which is more robust and better in terms of runtime and gave closer to the optimal result. This can be further improved in the future. Thus, I decided to build a testbed which is flexible and scalable enough that any solution approach of FCAPP can be tested on the same infrastructure using the same network configuration. This way, another benefit of comparing the evaluation results of different solution approach is achieved. The testbed provides a number of features such as run a real customized SDN controller, generate real network traffic, emulate data processing capability, control processing capacity of the virtual host, and control the bandwidth of the links and evaluate in real system time. These features are not easily achievable by means of a mathematical analysis or system level simulations. On the other hand, it provides a very flexible way to the users to test and evaluate their application which is difficult to obtain in a real testbed.

\section{Problem Definition}\label{sec:pd}
The main objectives of this thesis are to build an SDN-based emulation testbed; evaluate the functionality, feasibility, and performance of FCAPP solution; design the testbed in such a way that FCAPP solution could be easily adapted in the testbed; and fulfill the objective of FCAPP solution. Following are the high-level requirements:

\begin{itemize}
	\item Investigate and implement a suitable controller software to run in the testbed,
	\item Investigate suitable traffic models and implement it to emulate DFG creation,
	\item Investigate and implement a suitable way to emulate DFG processing,
	\item Design and implement the testbed using an emulation tool,
	\item Adapt FCAPP solution in the testbed, 
	\item Perform evaluation of FCAPP solution, and
	\item Analyze the testbed results. 
\end{itemize}

Different problems were solved to satisfy these requirements. The first problem is to select an emulation tool to create the testbed network with different network elements. The basic characteristics of an emulation environment to be considered when deciding on a tool are the following:
\begin{itemize}
	\item The ability to provide the same functionality as a real hardware network,
	\item The ability to create any kind of topology,
	\item Replicability and reproducibility of the experiment's setup,
	\item Capability to generate and capture real traffic, and
	\item Cost-effectiveness in terms of the hardware resource.
\end{itemize}
It is not an easy task since there is not much option available which support SDN (see Section \ref{sec:setnet}). Next is to adapt FCAPP solution in the emulation testbed, though the FCAPP solution algorithm was implemented in the simulation as part of a recent work \cite{7343600} (see Section \ref{sec:algoadap}). The simulation does not have to run real SDN controller or generate real network traffic or emulate data processing. These requirements are most important for the emulation and are not trivial to implement in the testbed.

At this point, it is important to understand FCAPP solution in high level. The main objective of FCAPP solution is to provide a complete control structure of all nodes in the network which means each node is satisfied by at least one Control Application (CA). This way, it provides the complete connection path of a node to its CA in the network. This path is not static; it might change while FCAPP solution algorithm will be executed next time. Therefore, the testbed needs an SDN controller to manage flow table in the switches/routers dynamically. A simple SDN controller is not enough for the testbed because the input for modification of the forwarding entries in the network would come from FCAPP solution. The SDN controller should have an interface to listen to FCAPP solution so that FCAPP solution can query the network controller for current network link details and request for updating the forwarding entries in the network as the way it wants. There is no ready-made controller available which could serve these specific requirements (see Section \ref{sec:fca}).

Another aspect of FCAPP solution is, it satisfies Data Flow Group (DFG), which DFG should be processed by which CA is determined by the solution. Each DFG has a particular bandwidth, latency, and lifetime (duration) and to satisfy the DFG, a CA requires a specific amount of processing capacity. To emulate this particular scenario, I need a real network traffic generator in the testbed and the traffic generation should be configurable and would have clear parametrised way of generating controlled traffic. It is also important to consider whether to use one centralized process to generate traffic for all DFG source nodes or distributed process running in each node to generate traffic for DFG (see Section \ref{sec:tmfg}).

Once traffic is generated in the network, it is necessary to implement a way to emulate the processing of data flow. Ideally, the CA in the network should capture the flow targeted to it, do some processing, and send it back to the flow source. This would require huge performance capability in the node where the CA is running. For an emulation testbed, that is not feasible due to the limitation of the hardware resource. Another option is to implement the processing time delay instead of actually perform the processing. Implementing the processing delay in a virtual SDN testbed is not easy and there is no ready-made solution available. Implementing the delay is not enough to fulfill the requirement of the data processing. The data flow has to come back to the network after the processing delay and back to the source of the flow. That means a regeneration of the traffic has to be implemented in the node where the CA is running (see Section \ref{sec:tmfg}).

While all aspects of the emulation are considered, at this stage, a laboratory setup design needs to be prepared, i.e., where to place which component of the emulation framework, the SDN controller, FCAPP solution, and traffic generator based on the available hardware resources at hand. Finally, the emulation runs in real system time, that means the execution had to happen in real-time as opposed to the simulation (see Section \ref{sec:emuexec}).

Next step is to evaluate FCAPP solution in the emulation testbed. For this purpose, I have to design test execution scenarios. There could be both functional or non-functional, i.e., performance, scenario. Then, I have to execute them and capture the results of the test execution. After capturing, I have to develop a way to parse the results so that it is easy to analyze and present the outcome.

\section{Related Work}
Researchers usually use simulation or emulation to test and evaluate new network protocols or applications. Recent works \cite{7343600} already evaluated FCAPP heuristic solution using simulation. One of the main requirements of this thesis is to evaluate the same in an emulation environment and the testbed should be implemented using an SDN-supported architecture. Existing options for setting up SDN-based testbed includes NS-3 \cite{ns-3}, Mininet \cite{Lantz:2010:NLR:1868447.1868466}, MaxiNet \cite{6857078}, EstiNet \cite{6588659}, DOT \cite{6838241} and OFnet \cite{ofnet}. FCAPP also provides a unique solution combining the control and data processing capability by the CA in the network. Considering all the aspects mentioned in problem definition (see Section \ref{sec:pd}), existing implementation which matches the similar requirements was diligently searched.

Several researchers have tried various techniques to set up a small-scale or large-scale network simulation or emulation testbeds. The authors of \cite{Gupta:2015:NRE:2756509.2756516} built an LTE testbed in an indoor lab network using NS-3 LTE LENA stack for the EU FP7 CROWD (Connectivity management for eneRgy Optimised Wireless Dense networks) project. They have modified the NS-3 MAC/PHY layer architecture to interface with LabView implementation of the LTE physical layer. They also had to modify the NS-3 core modules to enable real-time performance. They had a requirement of running an SDN controller in their testbed, that matches one of the testbed requirements in this thesis. However, the identified solution does not fulfill other requirements (e.g., data processing) for this thesis's purpose. Another drawback is they had to extend the NS-3 to have real-time operation support and to interface with hardware. 

In \cite{estinettestbed}, the authors used EstiNet to perform functional validation and performance evaluation of NOX OpenFlow controller. It was claimed that a creation of a network of thousands of simulated switch using EstiNet is possible. Though they created a small network to evaluate the functionality and performance of NOX controller, it has comparatively very limited requirements. Moreover, EstiNet is a proprietary emulation environment that is not a viable option in this case. 

In \cite{7503872}, the authors built a small, automated testbed using Mininet. Their goal was to implement an automated testbed for performance evaluation, and perform an in-depth analysis of different scheduling algorithms implemented in MPTCP (Multi-Path TCP) protocol. They used an SDN controller in their environment to setup flow paths in the network. Again, a small-scale testbed was created that do not have any requirements of data processing. 

In \cite{7726828}, the authors presented an SDN-based emulation of an academic networking testbed. They used Mininet and POX controller to build the testbed. They presented an emulation of their networking testbed called Point of Distribution (POD) used in the Internetworking Program (INWK). Their main requirement is to determine the capabilities and the limitations in the emulation with a comparison to their existing physical INWK PODs, which is more like checking the feasibility of INWK PODs in SDN than evaluating its performance. There was no testbed implementation that used MaxiNet, DOT or OFnet which has similar or matching requirement with this thesis.

In my knowledge, this work is a first where a large-scale distributed SDN-based emulation testbed was set up with a flexible option of using any SDN controller along with the capability of data processing in the network to test and evaluate a unique SDN solution proposed in \cite{7343600}.

\section{Structure of the Thesis}
This masters' thesis is structured as follows: Chapter \ref{BACK-CHAP} describes the background where different networking testing approaches were presented as a rationale in building an emulation testbed. It also presents a brief description of SDN and some of its elements and gave an introduction of FCAPP framework and the tools and technologies used in this thesis. 

In Chapter \ref{IMPL-CHAP}, the implementation details of the thesis work is fully laid out. In section \ref{sec:fca}, I describe about the SDN controller, how forwarding rules being populated in the switches and how it can be used for network discovery. In section \ref{sec:algoadap}, I describe how FCAPP solution works. Specific implementation necessary for emulation testbed setup and emulation execution flow are also extensively described in this chapter. 

In Chapter \ref{EVAL-CHAP}, testbed configuration and benchmarking are discussed to show performance parameters considered and assumed. It also includes experiment scenarios and the execution of those scenarios. Furthermore, a discussion about the execution result and comparison with the simulation result is included. 

Finally, Chapter \ref{CONC-CHAP} presents the findings and conclusions of this thesis and presents open problems and an outlook on possible future work. The Appendix \ref{APPD-CHAP} section describes the testbed infrastructure and its configuration along with installation instructions.




